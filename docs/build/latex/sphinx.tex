%% Generated by Sphinx.
\def\sphinxdocclass{report}
\documentclass[letterpaper,10pt,russian]{sphinxmanual}
\ifdefined\pdfpxdimen
   \let\sphinxpxdimen\pdfpxdimen\else\newdimen\sphinxpxdimen
\fi \sphinxpxdimen=.75bp\relax
\ifdefined\pdfimageresolution
    \pdfimageresolution= \numexpr \dimexpr1in\relax/\sphinxpxdimen\relax
\fi
%% let collapsible pdf bookmarks panel have high depth per default
\PassOptionsToPackage{bookmarksdepth=5}{hyperref}

\PassOptionsToPackage{booktabs}{sphinx}
\PassOptionsToPackage{colorrows}{sphinx}

\PassOptionsToPackage{warn}{textcomp}
\usepackage[utf8]{inputenc}
\ifdefined\DeclareUnicodeCharacter
% support both utf8 and utf8x syntaxes
  \ifdefined\DeclareUnicodeCharacterAsOptional
    \def\sphinxDUC#1{\DeclareUnicodeCharacter{"#1}}
  \else
    \let\sphinxDUC\DeclareUnicodeCharacter
  \fi
  \sphinxDUC{00A0}{\nobreakspace}
  \sphinxDUC{2500}{\sphinxunichar{2500}}
  \sphinxDUC{2502}{\sphinxunichar{2502}}
  \sphinxDUC{2514}{\sphinxunichar{2514}}
  \sphinxDUC{251C}{\sphinxunichar{251C}}
  \sphinxDUC{2572}{\textbackslash}
\fi
\usepackage{cmap}
\usepackage[T1]{fontenc}
\usepackage{amsmath,amssymb,amstext}
\usepackage{babel}





\usepackage[Sonny]{fncychap}
\ChNameVar{\Large\normalfont\sffamily}
\ChTitleVar{\Large\normalfont\sffamily}
\usepackage{sphinx}

\fvset{fontsize=auto}
\usepackage{geometry}


% Include hyperref last.
\usepackage{hyperref}
% Fix anchor placement for figures with captions.
\usepackage{hypcap}% it must be loaded after hyperref.
% Set up styles of URL: it should be placed after hyperref.
\urlstyle{same}

\addto\captionsrussian{\renewcommand{\contentsname}{Contents:}}

\usepackage{sphinxmessages}
\setcounter{tocdepth}{3}
\setcounter{secnumdepth}{3}


\title{Сайт для приюта "Лохматые судъбы"}
\date{февр. 07, 2025}
\release{08.02.2025}
\author{С.С.\@{} Александрова}
\newcommand{\sphinxlogo}{\vbox{}}
\renewcommand{\releasename}{Выпуск}
\makeindex
\begin{document}

\ifdefined\shorthandoff
  \ifnum\catcode`\=\string=\active\shorthandoff{=}\fi
  \ifnum\catcode`\"=\active\shorthandoff{"}\fi
\fi

\pagestyle{empty}
\sphinxmaketitle
\pagestyle{plain}
\sphinxtableofcontents
\pagestyle{normal}
\phantomsection\label{\detokenize{index::doc}}


\sphinxAtStartPar
Add your content using \sphinxcode{\sphinxupquote{reStructuredText}} syntax. See the
\sphinxhref{https://www.sphinx-doc.org/en/master/usage/restructuredtext/index.html}{reStructuredText}
documentation for details.
\subsubsection*{Modules}


\begin{savenotes}\sphinxattablestart
\sphinxthistablewithglobalstyle
\sphinxthistablewithnovlinesstyle
\centering
\begin{tabulary}{\linewidth}[t]{\X{1}{2}\X{1}{2}}
\sphinxtoprule
\sphinxtableatstartofbodyhook
\sphinxAtStartPar
{\hyperref[\detokenize{generated/main:module-main}]{\sphinxcrossref{\sphinxcode{\sphinxupquote{main}}}}}
&
\sphinxAtStartPar

\\
\sphinxhline
\sphinxAtStartPar
{\hyperref[\detokenize{generated/instance.convertor:module-instance.convertor}]{\sphinxcrossref{\sphinxcode{\sphinxupquote{instance.convertor}}}}}
&
\sphinxAtStartPar

\\
\sphinxbottomrule
\end{tabulary}
\sphinxtableafterendhook\par
\sphinxattableend\end{savenotes}

\sphinxstepscope


\chapter{main}
\label{\detokenize{generated/main:module-main}}\label{\detokenize{generated/main:main}}\label{\detokenize{generated/main::doc}}\index{module@\spxentry{module}!main@\spxentry{main}}\index{main@\spxentry{main}!module@\spxentry{module}}\subsubsection*{Functions}


\begin{savenotes}\sphinxattablestart
\sphinxthistablewithglobalstyle
\sphinxthistablewithnovlinesstyle
\centering
\begin{tabulary}{\linewidth}[t]{\X{1}{2}\X{1}{2}}
\sphinxtoprule
\sphinxtableatstartofbodyhook
\sphinxAtStartPar
\sphinxcode{\sphinxupquote{admin}}()
&
\sphinxAtStartPar
Страница редактирования записи о животном в приюте
\\
\sphinxhline
\sphinxAtStartPar
\sphinxcode{\sphinxupquote{animal\_detail}}(id)
&
\sphinxAtStartPar
Страница просмотра подробной информации о животном
\\
\sphinxhline
\sphinxAtStartPar
\sphinxcode{\sphinxupquote{close\_open\_list}}()
&
\sphinxAtStartPar
Функция сворачивания и разворачивания списка животных в приюте
\\
\sphinxhline
\sphinxAtStartPar
\sphinxcode{\sphinxupquote{create}}()
&
\sphinxAtStartPar
Страница создания новой записи о животном в приюте
\\
\sphinxhline
\sphinxAtStartPar
\sphinxcode{\sphinxupquote{delete}}(id)
&
\sphinxAtStartPar
Функия удаления записи о животном в приюте
\\
\sphinxhline
\sphinxAtStartPar
\sphinxcode{\sphinxupquote{edit}}(id)
&
\sphinxAtStartPar
Страница редактирования записи о животном в приюте
\\
\sphinxhline
\sphinxAtStartPar
\sphinxcode{\sphinxupquote{extract\_images}}(docx)
&
\sphinxAtStartPar
Функция извлечения изображений из документов word
\\
\sphinxhline
\sphinxAtStartPar
\sphinxcode{\sphinxupquote{extract\_text}}(docx)
&
\sphinxAtStartPar
Функция извлечения текста из документов word
\\
\sphinxhline
\sphinxAtStartPar
\sphinxcode{\sphinxupquote{import\_animals}}()
&
\sphinxAtStartPar
Функция импорта данных о животных в приюте из таблицы xlsx
\\
\sphinxhline
\sphinxAtStartPar
\sphinxcode{\sphinxupquote{import\_animals\_docx}}()
&
\sphinxAtStartPar
Функция импорта информации о животных в приюте из документов word
\\
\sphinxhline
\sphinxAtStartPar
\sphinxcode{\sphinxupquote{index}}()
&
\sphinxAtStartPar
Страница просмотра записей о животных в приюте
\\
\sphinxhline
\sphinxAtStartPar
\sphinxcode{\sphinxupquote{statistics}}()
&
\sphinxAtStartPar
Функция генерации графика тип животного\sphinxhyphen{}количество животных
\\
\sphinxhline
\sphinxAtStartPar
\sphinxcode{\sphinxupquote{yes\_or\_no}}(parametr)
&
\sphinxAtStartPar
Функция преобразования строк "да" и "нет" в соответствующие булевские значения
\\
\sphinxbottomrule
\end{tabulary}
\sphinxtableafterendhook\par
\sphinxattableend\end{savenotes}

\sphinxstepscope


\chapter{instance.convertor}
\label{\detokenize{generated/instance.convertor:module-instance.convertor}}\label{\detokenize{generated/instance.convertor:instance-convertor}}\label{\detokenize{generated/instance.convertor::doc}}\index{module@\spxentry{module}!instance.convertor@\spxentry{instance.convertor}}\index{instance.convertor@\spxentry{instance.convertor}!module@\spxentry{module}}\subsubsection*{Functions}


\begin{savenotes}\sphinxattablestart
\sphinxthistablewithglobalstyle
\sphinxthistablewithnovlinesstyle
\centering
\begin{tabulary}{\linewidth}[t]{\X{1}{2}\X{1}{2}}
\sphinxtoprule
\sphinxtableatstartofbodyhook
\sphinxAtStartPar
\sphinxcode{\sphinxupquote{yes\_or\_no}}(parametr)
&
\sphinxAtStartPar

\\
\sphinxbottomrule
\end{tabulary}
\sphinxtableafterendhook\par
\sphinxattableend\end{savenotes}


\chapter{Indices and tables}
\label{\detokenize{index:indices-and-tables}}\begin{itemize}
\item {} 
\sphinxAtStartPar
\DUrole{xref}{\DUrole{std}{\DUrole{std-ref}{genindex}}}

\item {} 
\sphinxAtStartPar
\DUrole{xref}{\DUrole{std}{\DUrole{std-ref}{modindex}}}

\item {} 
\sphinxAtStartPar
\DUrole{xref}{\DUrole{std}{\DUrole{std-ref}{search}}}

\end{itemize}


\renewcommand{\indexname}{Содержание модулей Python}
\begin{sphinxtheindex}
\let\bigletter\sphinxstyleindexlettergroup
\bigletter{i}
\item\relax\sphinxstyleindexentry{instance.convertor}\sphinxstyleindexpageref{generated/instance.convertor:\detokenize{module-instance.convertor}}
\indexspace
\bigletter{m}
\item\relax\sphinxstyleindexentry{main}\sphinxstyleindexpageref{generated/main:\detokenize{module-main}}
\end{sphinxtheindex}

\renewcommand{\indexname}{Алфавитный указатель}
\printindex
\end{document}